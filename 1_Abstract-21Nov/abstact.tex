\documentclass[11pt]{article}
\usepackage{geometry}
\geometry{a4paper, top=2.5cm, bottom=2.5cm, left=2.2cm, right=2.2cm}
\usepackage{fontspec}
\usepackage{xgreek}
\usepackage{graphicx}

\setmainfont{Baskerville}
\newfontfamily\sections{Futura}

\title{Wireless Body Area Network\\ Βιοϊατρική Τεχνολογία 2018}
\author{Φρανκ Μπλάννινγκ - 6698\\ Χριστίνα Θεοδωρίδου - 8055}
\date{\today}

\begin{document}

\maketitle

\subsection*{Περίληψη}
Ως αποτέλεσμα των πρόσφατων εξελίξεων στον τομέα της ηλεκτρονικής, των
ενσωματωμένων συστημάτων, της ασύρματης μετάδοσης ηλεκτρικής
ενέργειας, της ασύρματης επικοινωνίας, και της νανοτεχνολογίας είναι
πια εφικτή η ανάπτυξη συσκευών οι οποίες τοποθετούνται πάνω ή μέσα στο
σώμα και απαιτούν την ασύρματη μετάδοση πληροφοριών μεταξύ τους ή σε
ένα εξωτερικό σύστημα. Η έρευνα αυτού του τομέα ονομάζεται Wireless
Body Area Network (WBAN) και βασίζεται στα ακόμα αναπτυσσόμενα πρότυπα
IEEE 802.15.6 και IEEE 802.15.4j όπου σε συνδυασμό θέτουν την δομή
χρήσης των WBAN για ιατρική χρήση. Ο στόχος των WBAN είναι να
υποστηρίζουν ένα μεγάλο εύρος συσκευών και αισθητήρων οι οποίοι θα
έχουν πολύ χαμηλή κατανάλωση ενέργειας, θα έχουν μικρή απόσταση
εκπομπής και πολύ αξιόπιστη μετάδοση δεδομένων στην περιοχή εντός και
γύρω του σώματος. Υπάρχει μια μεγάλη γκάμα βιβλιογραφίας για την
αντιμετώπιση των προκλήσεων αυτών των εφαρμογών και στην παρούσα
εργασία θα παρουσιάσουμε την state-of-the-art κατάσταση ως έχει
σήμερα.


% WBAN supports a variety of real-time health monitoring and consumer
% electronics applications. The latest international standard for WBAN
% is the IEEE 802.15.6 standard which aims to provide an international
% standard for low power, short range, and extremely reliable wireless
% communication within the surrounding area of the human body,
% supporting a vast range of data rates for different applications


% Abstract—Recent developments and technological advance- ments in
% wireless communication, MicroElectroMechanical Sys- tems (MEMS)
% technology and integrated circuits has enabled low-power, intelligent,
% miniaturized, invasive/non-invasive micro and nano-technology sensor
% nodes strategically placed in or around the human body to be used in
% various applications, such as personal health monitoring.

% This exciting new area of research is called Wireless Body Area
% Networks (WBANs) and leverages the emerging IEEE 802.15.6 and IEEE
% 802.15.4j standards, specifically standardized for medical WBANs. The
% aim of WBANs is to simplify and improve speed, accuracy, and
% reliability of communication of sensors/actuators within, on, and in
% the immediate proximity of a human body. The vast scope of challenges
% associated with WBANs has led to numerous publications. In this paper,
% we survey the current state-of-art of WBANs based on the latest
% standards and publications. Open issues and challenges within each
% area are also explored as a source of inspiration towards Α
% developments in WBANs.


\end{document}
%%% Local Variables:
%%% mode: latex
%%% TeX-master: t
%%% TeX-engine: xetex

%%% End:
